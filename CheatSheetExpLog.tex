% Dale Roberts <dale.o.roberts@gmail.com> - Copyright (c) 2011-2014
\documentclass[twocolumn]{amsart}
\usepackage[pdftex, a4paper, margin=0.7cm, nohead, nofoot]{geometry}
\usepackage[utopia]{mathdesign}
\usepackage{amsmath}
\usepackage{amsfonts}
\usepackage{bbm}
\usepackage{setspace}
\usepackage{scalefnt}
\usepackage{microtype}
\usepackage[absolute]{textpos}
\usepackage[compact]{titlesec}

\renewcommand{\P}{\mathbf{P}}
\newcommand{\Q}{\mathbf{Q}}
\newcommand{\E}{\mathbf{E}}
\newcommand{\EP}{\mathbf{E}_\mathbf{P}}
\newcommand{\EQ}{\mathbf{E}_\mathbf{Q}}
\newcommand{\F}{\mathcal{F}}
\newcommand{\R}{\mathbf{R}}
\newcommand{\Normal}{\mathbf{N}}
\newcommand{\Cov}{\operatorname{\textbf{Cov}}}
\newcommand{\Call}{\mathcal{C}}
\newcommand{\Put}{\mathcal{P}}
\newcommand{\QV}[1]{\langle#1\rangle}
\newcommand{\sE}{\mathcal{E}}
\newcommand{\tW}{\widetilde W}
\setlength{\parskip}{0pt}
\setlength{\parindent}{0pt}
\setlength{\TPHorizModule}{30mm}
\setlength{\TPVertModule}{\TPHorizModule}
\textblockorigin{10mm}{10mm} % start everything near the top-left corner
\titleformat{\section}{\centering\selectfont\bf}{}{0em}{}

\def\parsedate #1:20#2#3#4#5#6#7#8\empty{#6#7/#4#5/20#2#3\parsetime#8\empty}
\def\parsetime #1#2#3#4#5\empty{ #1#2:#3#4}
\def\moddate#1{\expandafter\parsedate\pdffilemoddate{#1}\empty}


\begin{document}
\pagestyle{empty}
\thispagestyle{empty}
\setstretch{0.8}
\scalefont{0.8}

\begin{center}
\textbf{EXP/LOG APPLICATIONS CHEAT SHEET}
\vskip0.2em
\end{center}

\section*{Compound Interest}
\begin{itemize}
\item $\displaystyle P(t)=P_0\left (1+\frac{r}{n}\right )^{nt}$
\item $\displaystyle P(t)=P_0e^{rt}$
\end{itemize}

\section*{APR and APY}
\begin{itemize}
\item APR = annual interest rate
\item The APR does not take into account how often the interest is compounded.
\item APY = the simple interest that yields the same amount at the end of one year.
\item Knowing the APY makes it easier to shop around and find the best interest rate.
\item $\displaystyle \text{APY} = \left (1+\frac{{\text{APR}}}{n}\right )^{n}-1$
\item American Express sells 36-month CDs with APY=0.9\% and interest compounded daily and credited once a month. What is the APR? \emph{Answer:} Solve $\displaystyle 0.009=\left (1+\frac{{\text{APR}}}{365}\right )^{365}-1=0.00895985$.
\end{itemize}



\section*{Annuities}
\begin{itemize}
\item Annuity: a sum of money that is paid in regular equal payments.
\item Amount of an annuity: the sum of all the individual payments and all the interests. The amount is called $\displaystyle A_f$.
\item Assume that the time period in which the interest is compounded is equal to the time between payments.
\item $\displaystyle A_f=R\frac{(1+i)^n-1}{i}$, where $i$ is the interest rate PER TIME PERIOD, $n$ is the number of payments, $R$ is the regular annuity payment.
\item Example: find the amount of annuity that consists of 20 semiannual payments at \$5000 each into an account that pays an interest of 12\% per year. \emph{Answer:} $\displaystyle R=5000, t=\frac{0.12}{2}, n=20$ and use the formula for $A_f$.
\end{itemize}

\section*{Exponential Growth}
\begin{itemize}
\item $\displaystyle n(t)=n_0\cdot2^{\frac{t}{a}}$, where $a$ is the doubling time;
%\item $\displaystyle n(t)=n_0\cdot 2{\frac{t}{a}}$ ($a$ is the doubling time};
\item $\displaystyle n(t)=n_0\cdot e^{rt}$ ($r$ is the relative growth rate).
\end{itemize}

\section*{Credit Card APR}
Example: Wells Fargo Cash Back Visa Signature Card. For this credit card:
\begin{itemize}
\item  APR is 12.5\% to 25.99\%, based on your credit worthiness.
\item Balance is calculated with the \emph{average daily balance method}. Average daily balance:  (sum of the daily balance in the billing cycle) divided by (number of days in the billing cycle).
\item Quoted from the WellsFargo website $\rightarrow$ \emph{Daily Balance Computation Method. We use the daily balance method to calculate the interest on your account. This method applies a daily periodic rate to the principal balance in the account each day.} Meaning: the interest is compounded daily.
\item Assume a customer has an APR of 15\% and an average daily balance of \$500 in a billing cycle that is 28 days long. What is the customer's interest charge for that billing cycle? \emph{Answer:} $\displaystyle 28\cdot500\cdot \left ( \frac{0.15}{365}\right )=5.75$. The interest charge for that month is \$5.75.
\item Note: with the average daily balance method, the time of your payment and charges matter!
\end{itemize}

\vfill
% \hline
\begin{center}
{
\bf
\scalefont{0.75}
\vskip0.15em
\textbf{\today}
\vskip0.15em
}
% \hline

\end{center}

\end{document}