\documentclass[11pt,answers]{exam}

\usepackage{etex}
\usepackage{amssymb,amsmath,multicol} %<-- InWorksheetExam1 i also have fancyhdr,

\usepackage[metapost]{mfpic}
\usepackage[pdftex]{graphicx}

\usepackage{pst-plot}
\usepackage{pgfplots}
\pgfplotsset{compat=1.9}

\usepackage{tikz}
\usepackage{tkz-2d}
\usepackage{tkz-base}
\usetikzlibrary{calc}

\usepackage[inline]{enumitem}
\usepackage{refcount}%<-- non in WorksheetExam1

%\renewcommand{\headrulewidth}{0pt}

\newcommand{\vasymptote}[2][]{
    \draw [densely dashed,#1] ({rel axis cs:0,0} -| {axis cs:#2,0}) -- ({rel axis cs:0,1} -| {axis cs:#2,0});
}

\addpoints
%\printanswers
\noprintanswers

\opengraphsfile{Q5a_Spring_15}

\begin{document}
\extrawidth{-0.3in}
\pagestyle{headandfoot}

\setlength{\hoffset}{-.25in}

\extraheadheight{-.4in}
\runningheadrule
\firstpageheader{\bfseries {MATH1-UC 1171}}{ \bfseries {Quiz 5 }}{\bfseries {3/31/2015}} 



\firstpagefooter{} %%&&CHANGED
                {}
                {Points earned: \hbox to 0.5in{\hrulefill}
                 out of  \pointsonpage{\thepage} points}
                 
						

\vspace*{0.7cm}
\hbox to \textwidth { \scshape {Name:} \enspace\hrulefill}
\vspace{0.2in}




\pointpoints{point}{points}

\begin{questions}


\addpoints

\question[1] Is the following statement true or false? A rational function may have two distinct horizontal asymptotes.
\begin{oneparchoices}
\choice True \choice False
\end{oneparchoices}
\question[1] If a rational function has a horizontal asymptote, then the graph of the function can never cross the horizontal asymptote.
\begin{oneparchoices}
\choice True
\choice False
\end{oneparchoices}
\question[1] If a rational function has a vertical asymptote, then the graph of the function can never cross the vertical asymptote.
\begin{oneparchoices}
\choice True
\choice False
\end{oneparchoices}
\question[1] What are the $x$-intercepts of $\displaystyle f(x)=\frac{(x-2)(x+3)}{(x-1)(x+2)(x-5)}$?

\begin{oneparchoices}
\choice $(2,0),(-3,0)$
\choice $(-2,0),(3,0)$
\choice None of these.
\end{oneparchoices}
\question[1] When $x\to \infty$ then:

\begin{oneparchoices}
\choice  $\displaystyle \frac{2x(x+3)}{(x-2)(x-5)}\to 2$
\choice  $\displaystyle \frac{2x(x+3)}{(x-2)(x-5)}\to 0$
\choice  $\displaystyle \frac{2x(x+3)}{(x-2)(x-5)}\to \infty$
\end{oneparchoices}
\question[3] (Explanations are required.) The graph shown below is the graph of one of these two rational functions.

\begin{itemize}
\item $\displaystyle f(x)=\frac{x}{(x+1)^2(x-1)}$
\item $\displaystyle g(x)=\frac{x}{(x+1)(x-1)^2}$
\end{itemize}
Which is the function, and how do you know? 

%%%%%%%%%%%%%%%%%%%%%%%%%%
\begin{mfpic}[30]{-3}{3}{-2.5}{2.5}
\arrow \reverse \arrow \function{-3, -1.5, 0.1}{(x)/((x+1)*(x+1)*(x-1))}
\arrow \reverse \arrow \function{-0.6, 0.8, 0.1}{(x)/((x+1)*(x+1)*(x-1))}
\arrow \reverse \arrow \function{1.1, 2.5, 0.1}{(x)/((x+1)*(x+1)*(x-1))}
%\point[3pt]{(0,0)}
\dashed \polyline{(-1,-2), (-1,2.5)}
\dashed \polyline{(1,-2), (1,2.5)}
\tlabel[cc](3,-0.3){\scriptsize $x$}
\tlabel[cc](0.2,2.8){\scriptsize $y$}
\axes
\xmarks{-2 step 0.5 until 2}
\ymarks{-2 step 0.5 until 2}
\tiny
\tlpointsep{4pt}
 \axislabels {x}{   {$-2\hspace{7pt}$} -2,  {$2$} 2}
\axislabels {y}{  {$-2$} -2,{$-1$} -1, {$1$} 1, {$2$} 2}
\normalsize
\end{mfpic}

\fillwithdottedlines{1.3in}
%%%%%%%%%%%%%%%%%%%%%%%%%%



\end{questions}

\end{document}                 