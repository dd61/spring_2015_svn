\documentclass[11pt,answers]{exam}

\usepackage{etex}
\usepackage{amssymb,amsmath,multicol} %<-- InWorksheetExam1 i also have fancyhdr,

\usepackage[metapost]{mfpic}
\usepackage[pdftex]{graphicx}

\usepackage{pst-plot}
\usepackage{pgfplots}
\pgfplotsset{compat=1.9}

\usepackage{tikz}
\usepackage{tkz-2d}
\usepackage{tkz-base}
\usetikzlibrary{calc}

\usepackage[inline]{enumitem}
\usepackage{refcount}%<-- non in WorksheetExam1

%\renewcommand{\headrulewidth}{0pt}

\newcommand{\vasymptote}[2][]{
    \draw [densely dashed,#1] ({rel axis cs:0,0} -| {axis cs:#2,0}) -- ({rel axis cs:0,1} -| {axis cs:#2,0});
}

\addpoints
%\printanswers
\noprintanswers

\opengraphsfile{Q4a_Fall_15}

\begin{document}
\extrawidth{-0.3in}
\pagestyle{headandfoot}

\setlength{\hoffset}{-.25in}

\extraheadheight{-.3in}
\runningheadrule
\firstpageheader{\bfseries {MATH1-UC 1171}}{ \bfseries {Quiz 4 }}{\bfseries {10/6/2015}} 



\firstpagefooter{} %%&&CHANGED
                {}
                {Points earned: \hbox to 0.5in{\hrulefill}
                 out of  \pointsonpage{\thepage} points}
                 
						

\vspace*{0.1cm}
\hbox to \textwidth { \scshape {Name:} \enspace\hrulefill}
\vspace{0.1in}




\pointpoints{point}{points}

\begin{questions}


\addpoints

\question The function $f(x)$ is given by the formula $\displaystyle f(x)=x^2$, and the domain is the interval $[-4,0]$

\begin{parts}
\bonuspart[2] Write the range of $f(x)$ in interval form. \dotfill 

\part[3] Graph the function (note: the domain is $[-4,0]$) and use your graph to explain why this function has an inverse.

\begin{center}

\begin{mfpic}[20]{-6}{6}{-2}{5}

%\polyline{(0,-2), (4,1), (4,2), (5,3)}

%\polyline{(0,1), (2,0)} 

%\polyline{(2,0), (4,2)}

%\point[5pt]{(0,1), (4,2)}

\axes

\xmarks{-5,-4,-3,-2,-1,1,2,3,4,5}

\ymarks{-2,-1,1,2,3,4,}

\tlpointsep{4pt}

\axislabels {x}{{\tiny $1$} 1, {\tiny $2$} 2, {\tiny $3$} 3, {\tiny $4$} 4, {\tiny $5$} 5, {\tiny $-1$} -1, {\tiny $-2$} -2, {\tiny $-3$} -3, {\tiny $-4$} -4, {\tiny $-5$} -5}

\axislabels {y}{{\tiny $1$} 1,{\tiny $2$} 2, {\tiny $3$} 3, {\tiny $4$} 4,  {\tiny $-1$} -1, {\tiny $-2$} -2}

  % Grid
  %\drawcolor[gray]{0.25}
  %\gridlines{1, 1}
\drawcolor[gray]{0.75} 
\grid{1,1}

\end{mfpic}

\end{center}

\fillwithdottedlines{2.5cm}

\part[2] Find $\displaystyle f^{-1}$ and show your work step by step.

\fillwithdottedlines{2.5cm}

\part[2] Write the range of $\displaystyle f^{-1}$ in interval form. \dotfill 


\end{parts}
\question[1] For his services, a private investigator requires a \$500 retention fee plus \$95 per hour. The number of hours the investigator spends working on a case is $x$, and the
investigator's fee as a function of $x$ is $f(x)$. What does 
$\displaystyle f^{-1}$ represent? Circle the {\underline{one}} best answer.

\begin{choices}
\choice $\displaystyle f^{-1}$ represents the amount in dollars paid for $x$ numbers of hours of investigation.
 
\choice $\displaystyle f^{-1}$ represents the amount in dollars paid beyond the retention fee for $x$ numbers of hours of investigation.
    

\choice $\displaystyle f^{-1}$ represents the amount paid per each hour of investigation.
\choice $\displaystyle f^{-1}$ represents the number of hours of investigation the investigator spends on a case for $x$ dollars.
 
\choice $\displaystyle f^{-1}$ represents the maximum number of hours of investigation the investigator will spend on a case.
 

\end{choices}

\end{questions}

\end{document}                 